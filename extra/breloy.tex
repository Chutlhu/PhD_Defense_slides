% Some commands for the template
% -------------------------------------------------------------------------------------

\setbeamertemplate{section in toc}[circle]
\setbeamerfont{section in toc}{series=\bf}
\setbeamertemplate{subsection in toc}[ball unnumbered]
\setbeamerfont{subsection in toc}{size=\footnotesize}


% Navigation circles
\makeatletter
\setbeamertemplate{headline}{%

\begin{beamercolorbox}[colsep=1.5pt]{upper separation line head}
\end{beamercolorbox}

\begin{beamercolorbox}{section in head/foot}
  \vskip2pt\insertnavigation{\paperwidth}\vskip2pt
\end{beamercolorbox}%

\begin{beamercolorbox}[colsep=1.5pt]{lower separation line head}
\end{beamercolorbox}
}
\setbeamercolor{section in head/foot}{fg=normal text.bg, bg=structure.fg}

% Navigation circles become squares
\setbeamertemplate{mini frame}
{%
    \begin{pgfpicture}{0pt}{0pt}{.1cm}{.1cm}
        \pgfpathrectangle{\pgfpointorigin}{\pgfpoint{.7*\the\beamer@boxsize}{.7*\the\beamer@boxsize}}
        \pgfusepath{fill,stroke}
    \end{pgfpicture}%
}

\setbeamertemplate{mini frame in current subsection}
{%
    \begin{pgfpicture}{0pt}{0pt}{.1cm}{.1cm}
        \pgfpathrectangle{\pgfpointorigin}{\pgfpoint{.7*\the\beamer@boxsize}{.7*\the\beamer@boxsize}}
        \pgfusepath{stroke}
    \end{pgfpicture}%
}

\makeatother

% Some personal commands
\newcommand{\persoframetitle}[1]{
  \vspace{1em}
  {\bf \Large #1}
  \vspace{0.3em}
  \hrule
  \vfill
}

% The progress bar
% -------------------------------------------------------------------------------------
\usenavigationsymbolstemplate{}
\definecolor{pbblue}{HTML}{000000}% color for the progress bar and the circle

\makeatletter

\def\progressbar@progressbar{} % the progress bar

\newcount\progressbar@tmpcounta% auxiliary counter

\newcount\progressbar@tmpcountb% auxiliary counter

\newdimen\progressbar@pbht %progressbar height

\newdimen\progressbar@pbwd %progressbar width

\newdimen\progressbar@rcircle % radius for the circle

\newdimen\progressbar@tmpdim % auxiliary dimension

\progressbar@pbwd=\linewidth

\progressbar@pbht=1pt

\progressbar@rcircle=2.5pt

\def\progressbar@progressbar{%

    \progressbar@tmpcounta=\insertframenumber

    \progressbar@tmpcountb=\inserttotalframenumber

    \progressbar@tmpdim=\progressbar@pbwd

    \multiply\progressbar@tmpdim by \progressbar@tmpcounta

    \divide\progressbar@tmpdim by \progressbar@tmpcountb

    \begin{tikzpicture}

      \draw[line width=\progressbar@pbht]

      (0pt, 0pt) -- ++ (\progressbar@pbwd,0pt);

      \filldraw %
      (\the\dimexpr\progressbar@tmpdim-\progressbar@rcircle\relax, .5\progressbar@pbht) circle (\progressbar@rcircle);

      \node[draw,text width=3.5em,align=center,inner sep=1pt,anchor=east] at (0,0) {\insertframenumber/\inserttotalframenumber};

    \end{tikzpicture}%

}

%% Une progressbar / logo dans le bas des slides

%\addtobeamertemplate{footline}{}

%{%

%	\begin{beamercolorbox}[wd=\paperwidth,ht=4ex,center,dp=1ex]{white}%

%		\progressbar@progressbar%

%	\end{beamercolorbox}%11

%\setlength\unitlength{1ex}%

%\begin{picture}(0,0)

% \put{} defines the position of the frame

%\put(80,4){International Geoscience and Remote Sensing Symposium 30/07/2019}%

%\put(135,4){\includegraphics[width=0.15\textwidth]{Figures/logo_igarss_dark.pdf}}%

%\put(110,4){Ammar Mian, 30 July 2019}%

%\end{picture}%

%}



% Beamer: How to place images behind text (z-order)

% (http://tex.stackexchange.com/a/134311)

% -------------------------------------------------------------------------------------

\makeatletter

\newbox\@backgroundblock

\newenvironment{backgroundblock}[2]{%

  \global\setbox\@backgroundblock=\vbox\bgroup%

    \unvbox\@backgroundblock%

    \vbox to0pt\bgroup\vskip#2\hbox to0pt\bgroup\hskip#1\relax%

}{\egroup\egroup\egroup}

\addtobeamertemplate{background}{\box\@backgroundblock}{}

\makeatother

