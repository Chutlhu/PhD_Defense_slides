\subsection{Motivation}

\begin{frame}{Scenario}

    Meeting - 2020 edition

    \vspace{1em}
    \begin{columns}[onlytextwidth]
        \begin{column}{0.69\textwidth}
            \begin{figure}
                \includegraphics[width=\textwidth]{example-image}
            \end{figure}
        \end{column}
        \begin{column}{0.29\textwidth}
            \textbf{Sound}
            \begin{itemize}
                \item produced by \alert{sources}
                \item recorded by \alert{microphones}
                \item corrupted by \alert{noise}
                \item  propagates in the \alert{room}
                    \\$\hookrightarrow$ \alert{reverberation}
            \end{itemize}
        \end{column}
    \end{columns}

    \vfill
    \textcolor{gray}{Attention: artificial sound vs \textbf{(natural) microphone recordings}}

\end{frame}

\begin{frame}[t]{Echo-aware signal processing for \alert{audio scene analysis}}

    \begin{columns}
        \begin{column}[t]{0.3\textwidth}
            \centering
            \alert{Semantic} information

            \vspace*{0.5em}
            \includegraphics[width=\textwidth]{example-image-a}
            about source nature and semantic content
        \end{column}
        \begin{column}[t]{0.3\textwidth}
            \centering
            \alert{Spatial} information

            \vspace*{0.5em}
            \includegraphics[width=\textwidth]{example-image-b}
            about source position and room geometry
        \end{column}
        \begin{column}[t]{0.3\textwidth}
            \centering
            \alert{Temporal} information

            \vspace*{0.5em}
            \includegraphics[width=\textwidth]{example-image-c}
            about events activity
        \end{column}
    \end{columns}

    \vfill
    \begin{mydefblock}{Audio Scene Analysis}
        Extraction and organization of all the information in the sound
    \end{mydefblock}

    \vfill
    \begin{figure}
        \centering
        \includegraphics[width=0.2\textwidth]{example-image-a}
        \hfill
        $\rightarrow$
        \hfill
        \includegraphics[width=0.2\textwidth]{example-image-c}
        \hfill
        $\rightarrow$
        \hfill
        \includegraphics[width=0.2\textwidth]{example-image}
    \end{figure}

    Animals do it, Humans do it. Can computer and robots do it?

\end{frame}

\begin{frame}[t]{Echo-aware \alert{signal processing for audio scene analysis}}

    \begin{mydefblock}{Signal Processing}
        Mathematical models, frameworks and tools to tackle and solve such problems
    \end{mydefblock}

    \vfill
    \begin{columns}[onlytextwidth]
        \begin{column}{0.55\textwidth}
            \begin{figure}
                \includegraphics[width=\textwidth]{example-image-a}
            \end{figure}
        \end{column}\hfill
        \begin{column}{0.42\textwidth}
            Some problems\hfill\tikzmark{right}
            \begin{itemize}
                \item Speaker Verification\tikzmark{top1}\tikzmark{bot1}
                \item Sound Source Separation \tikzmark{top2}
                \item Speech Enhancement
                \item Automatic Speech Recognition \tikzmark{bot2}
                \item Sound Source Localization
                \item Room Geometry Estimation
                \item Voice Activity Detection
                \item Diarization
                \item RT$_{60}$ estimation
                \item Wall Absorption Estimation
                \item \textit{and many many other}
            \end{itemize}

            \begin{tikzpicture}[overlay, remember picture]
                \node[anchor=base] (a) at (pic cs:top1) {\vphantom{h}}; % push the mark to the top of the line (ie including ascenders)
                \node[anchor=base] (b) at (pic cs:bot1) {\vphantom{g}}; % push the mark to the bottom of the line (ie including descenders)
                \draw [decoration={brace,amplitude=0.5em},decorate,thick,gray]
                 (a.north -| {pic cs:right}) -- node[right,inner sep=1em] {Who?} (b.south -| {pic cs:right});
            \end{tikzpicture}
            \begin{tikzpicture}[overlay, remember picture]
                \node[anchor=base] (a) at (pic cs:top2) {\vphantom{h}}; % push the mark to the top of the line (ie including ascenders)
                \node[anchor=base] (b) at (pic cs:bot2) {\vphantom{g}}; % push the mark to the bottom of the line (ie including descenders)
                \draw [decoration={brace,amplitude=0.5em},decorate,thick,gray]
                 (a.north -| {pic cs:right}) -- node[right,inner sep=1em] {What?} (b.south -| {pic cs:right});
            \end{tikzpicture}
        \end{column}\hfill
    \end{columns}
    % \pause
    % Also known as auditory scene analysis or computer auditory scene analysis.
    % \\Inverse and Forward problems
    % \\Blind and Informed problems

    \vfill
    \begin{block}{Everything is connected}
        HOW  $\rightarrow$ WHERE $\rightarrow$ WHEN $\rightarrow$ WHAT
    \end{block}

\end{frame}


% \begin{frame}{Echo-aware \alert{signal processing} for audio scene analysis}

%     \vfill
%     \begin{block}{Signal Model}
%         \begin{columns}
%             \begin{column}{0.45\textwidth}
%                 \centering
%                 \includegraphics[width=0.5\textwidth]{example-image-a}
%             \end{column}
%             \begin{column}{0.58\textwidth}
%                 \begin{itemize}
%                     \item produced by sources
%                     \item propagates in the room
%                     \item corrupted by noise
%                     \item recorded by microphone
%                 \end{itemize}
%             \end{column}
%         \end{columns}
%     \end{block}

% \end{frame}

\begin{frame}[t]{\alert{Echo-aware signal processing for audio scene analysis}}

    \begin{mydefblock}{Acoustic Echoes}
        \begin{itemize}
            \item Sound reflection standing out for time and strength w.r.t. to the total reverberation
            \item repetition of the source sound
            \item but later
            \item both outdoor and indoor
        \end{itemize}
    \end{mydefblock}

    \vfill
    \begin{block}{Echo-aware processing}
    \end{block}

    \begin{columns}[T,onlytextwidth]
        \begin{column}{0.3\textwidth}
            \alert{Free-field} processing
            \\{\small(only direct path)}
        \end{column}\hfill
        \begin{column}{0.3\textwidth}
            \alert{Echo} processing
            \\{\small(only specular reflection)}
        \end{column}\hfill
        \begin{column}{0.3\textwidth}
            \alert{Reverberant} processing
            \\{\small(entire sound field)}
        \end{column}
    \end{columns}%

    \vfill
    \begin{columns}[T,onlytextwidth]
        \begin{column}{0.3\textwidth}
            \small
            \begin{itemize}
                \item[\cmark] \textcolor{mygreen}{``simple'' processing} %closed-from formulas %fully deterministic
                \item[\cmark] \textcolor{mygreen}{``short'' processing} % STFT, no latency
                \item[\xmark] \textcolor{myred}{Reflection are interferences}
                \item[\xmark] \textcolor{myred}{Neglect most of the sound energy}
                \item[\xmark] \textcolor{myred}{incoherence $\implies$ distortion}
            \end{itemize}
        \end{column}\hfill
        \begin{column}{0.3\textwidth}
            \small
            \begin{itemize}
                \item short processing %  echoes $\rightleftarrows$ positions
                \item simple processing
                \item reflection are integrated,
                \\late reverberation can considered
                \item reflection are difficult to estimate
                \item reflection need to be correctly estimated
            \end{itemize}
        \end{column}\hfill
        \begin{column}{0.3\textwidth}
            \small
            \begin{itemize}
                % fully deterministic and stochastic processing
                \item difficult processing %estimation is extremely difficult, more parameters to estimate
                \item long precessing
                \item Reflection perfectly integrated
            \end{itemize}
        \end{column}
    \end{columns}

    %     \\Anechoic processing
    %     \begin{itemize}
    %         \item short processing
    %         \item[\cmark] sound field tends to be diffuse
    %         \item[\xmark] sound reflection as interferences
    %         \\neglect most of the sound energy
    %         \\ignore the correlation between the direct sound and its reflection and consequently may result in a distorted output
    %         \item[\xmark] coherent processing becomes impossible
    %     \end{itemize}
    %     Echo-processing
    %     \begin{itemize}
    %         \item Middle processing
    %         perceive coming from all directions~\cite{daldegan1988}
    %     \end{itemize}
    %     Reverberant processing
    %     \begin{itemize}
    %         \item Long processing
    %         \item[\cmark] narrowband approximation is valid
    %         \item[\cmark] sound field is described by full RIRs
    %         \item[\cmark] all reflection can be \alert{coherently} processed
    %         \item[\xmark] long frames impose lantency
    %     \end{itemize}
    % \end{block}

\end{frame}

\subsection{Outline}

\begin{frame}[t]{Thesis goal and contribution}

    \vfill
    \begin{columns}[onlytextwidth]
        \begin{column}[T]{0.3\linewidth}
            \centering
            \textbf{Audio Scene Analysis}
            \\\downarrow
            \\context and problems
        \end{column}\hfill
        \begin{column}[T]{0.3\linewidth}
            \centering
            \textbf{Signal Processing}
            \\\downarrow
            \\models and frameworks
        \end{column}\hfill
        \begin{column}[T]{0.3\linewidth}
            \centering
            \textbf{Acoustic Echoes}
            \\\downarrow
            \\better processing
        \end{column}\hfill
    \end{columns}

    \vfill
    \begin{mydefblock}{Thesis objective}
        \begin{enumerate}
        \item provide new methodologies and data to process and estimate acoustic echoes
        \item Turning echoes into friends \cite{Riberio}
        \\extend previous classical methods for audio scene analysis
        \end{enumerate}
    \end{mydefblock}

    \vfill
    \begin{block}{contribution}
        Estimation
        \begin{itemize}
            \item Knowledge-based echo estimation, aka \blaster
            \item Learning-based echo estimation, aka \lantern
        \end{itemize}
        Application
        \begin{itemize}
            \item Echo-aware Source Separation, aka \separake
            \item Echo-aware Source Localization, aka \mirage
            \item Echo-aware Speech Enhancement
            \item Echo-aware Room Geometry Estimation
        \end{itemize}
        Data
        \begin{itemize}
            \item a Echo-aware database, aka \dechorate
        \end{itemize}
    \end{block}

\end{frame}

% change color of Beamer subsection
% https://tex.stackexchange.com/questions/356237/changing-text-color-in-subsection-in-tableofcontents-in-beamer
\begin{frame}[standout]{1D Outline}
    \begin{columns}
        \begin{column}{0.35\textwidth}
            Echo-aware signal processing
            \\for audio scene analysis
        \end{column}
        \begin{column}{0.05\textwidth}
        \end{column}
        \begin{column}{0.6\textwidth}
            {
                \setbeamercolor{part in toc}{fg=gray}
                \setbeamercolor{section in toc}{fg=gray}
                \setbeamercolor{subsection in toc}{fg=white}
                \tableofcontents
            }
        \end{column}
    \end{columns}
\end{frame}