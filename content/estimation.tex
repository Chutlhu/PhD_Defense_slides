\subsection*{State of The Art}

\begin{frame}[t]{Acoustic Echo Retrieval \hfill\faBook}
    \begin{columns}[T,onlytextwidth]
        \column{0.55\textwidth}
            Estimating early (strong) reflections for microphones recordings, i.e.,
            \begin{equation*}
                \{\contMic_i\}_i \longrightarrow \{ \tauir, \textcolor{gray}{\ampir} \}_{i,r}
            \end{equation*}
        \column{0.4\textwidth}
            \begin{figure}
                \centering
                \includegraphics[width=\textwidth]{./figures/arrivals.png}
            \end{figure}
    \end{columns}

    \vfill
    \begin{columns}[T,onlytextwidth]
        \column{0.48\textwidth}
        \textbf{2 Scenarios:} the source signal is

        \begin{center}
            \includegraphics[width=.9\textwidth]{./figures/active-passive.png}
        \end{center}

        \only<4>{
        \column{0.48\textwidth}
        Active
        \vspace{-2mm}
        \begin{itemize}
            \small
            \item[\faEye] \textbf{non-blind} problem
            \item[\faVolumeUp] \textbf{intrusive} or specific setups
            % \item known emitted signal
            % \item Time of Arrival (\textbf{TOA}s) accessible
            % \\\hspace{.3em} $\implies$ \textbf{single} mic
            \\\hspace{-1em}\addendum{\footnotesize \textbf{Application:} sonar, calibration, measurements, etc.}
        \end{itemize}

        Passive
        \vspace{-2mm}
        \begin{itemize}
            \small
            \item[\faEyeSlash] \textbf{blind inverse} problem (harder)
            \item[\faMicrophone] \textbf{passive} and more common setups
            % \item \alert{un}knwon
            % \item Time Difference of Arrivals (\textbf{TDOA}s) only
            % \\\hspace{.3em} \textbf{multi-mic}
            \\\hspace{-1em}\addendum{\footnotesize \textbf{Applications:} recording on smart speakers, laptop, etc.}
        \end{itemize}
        }
    \end{columns}

    \vfill
    \textcolor{myred}{\textbf{Our case:} signal source and passive system of ($I$ microphones)}

\end{frame}

\begin{frame}[t]{\alert{Passive} Acoustic Echo Retrieval \hfill\faBook}

        \vspace{.5em}
        \begin{columns}[T,onlytextwidth] % align columns
            \begin{column}{.48\textwidth}
                \textbf{RIR-\alert{based} approaches}
            \end{column}
            \begin{column}{.48\textwidth}
                \textbf{RIR-\alert{agnostic} approaches}
            \end{column}%
        \end{columns}

        \vspace{.5em}
        \begin{columns}[onlytextwidth] % align columns
            \begin{column}{.48\textwidth}
                \includegraphics[trim={0 31em 0 7em},clip,width=.9\textwidth]{./figures/based-agnostic.png}
            \end{column}
            \begin{column}{.48\textwidth}
                \includegraphics[trim={0 0 0 40em},clip,width=.9\textwidth]{./figures/based-agnostic.png}
            \end{column}%
        \end{columns}

        \vspace{.5em}
        \begin{columns}[T,onlytextwidth] % align columns
            \begin{column}{.48\textwidth}
                \small
                \begin{enumerate}
                    \item Discrete optimization $\implies$ RIRs
                    \item Peak picking $\implies$ Echoes
                \end{enumerate}
            \end{column}
            \begin{column}{.48\textwidth}
                \small
                \begin{enumerate}
                    \item Direct off-grid estimation of $\set{\tauir, \ampir}$
                    e.g., with maximum-likelihood
                    % \\\textcolor{gray}{(+ \small direction of arrivals can be used instead)}
                \end{enumerate}
            \end{column}%
        \end{columns}

        \vspace{1em}
        \begin{columns}[T,onlytextwidth] % align columns
            \column{.48\textwidth}
                \small
                \begin{itemize}
                    \item[\cmark] \pro{BCE is well and known studied}
                    \item[\cmark] \pro{reasonably good for some application}
                    \\{\scriptsize~\cite{crocco2016estimation}}
                \end{itemize}
            \column{.48\textwidth}
                \small
                \begin{itemize}
                    \item[\cmark] \pro{No full RIRs \& no peak picking}
                    \begin{itemize}
                        \item[$\to$] lower complexity
                        \item[$\to$] less hyperparameters
                    \end{itemize}
                    \end{itemize}
        \end{columns}

        \vspace{1em}
        \begin{columns}[T,onlytextwidth] % align columns
            \small
            \column{.48\textwidth}
                \begin{itemize}
                    \item[\xmark] \con{Full RIRs need to be estimated}
                    \item[\xmark] \con{Peak picking has hyperparameters}
                    \item[\xmark] \con{Issues due to \textit{discrete} estimation}
                \end{itemize}
            \column{.48\textwidth}
                \begin{itemize}
                    \item[\xmark] \con{exploratory \faWpexplorer}
                    \\\addendum{no standard solver, few works on audio}
                \end{itemize}


                \textcolor{myred}{\textbf{Proposed appoach}} RIR-agnostic \& continuous:
                \begin{mycontriblock}
                    \begin{enumerate}
                        \item Learning-based approach
                        \item Analytical approach
                    \end{enumerate}
                \end{mycontriblock}
        \end{columns}

\end{frame}

\subsection{\lantern}

\begin{frame}{Proposed approach: learning-based \& off-grid \hfill\faProjectDiagram}

    % \textbf{Recall}: AER $\Leftrightarrow$ $\{\contMic_i\}_i \overset{?}{\longrightarrow} \{ \tauir, \textcolor{black!20}{\ampir} \}_{i,r}$
    \begin{mydefblock}{Idea: (Deep) Learning-based AER}
        \begin{enumerate}
            \item Use \alert{virtually} supervised deep learning models
            \item Estimate first echo (simple but important) \addendum{\footnotesize \faReply~See Section Application}
            \item Only 2 microphones attending 1 source
        \end{enumerate}
    \end{mydefblock}

    \vfill
    \textbf{Motivations:}
    \begin{itemize}
        \item This \textit{inverse} mapping is difficult has no solution in general, \textit{direct} mapping ``is not''
        \\$\rightarrow$ acoustic simulators:
            $\text{mic/src/room geometry}
            \; \longrightarrow \;
            \set{\tauir,\ampir}, \; \contRIR_i, \quad \contMic_i$
        \item Acoustic simulator are ``simple'', versatile and fast
        \\$\to$ large dataset
        % \item This approach is successful in \textit{Sound Source Localization}
        % \\$\to$ position is related to echoes
        % \\{\small\cite{kataria2017hearing,nguyen2018autonomous,perotin2019regression} \textcolor{myred}{\faExclamationTriangle~Not only DNN}}
    \end{itemize}

\end{frame}

\begin{frame}{Proposed approach: models \hfill\faProjectDiagram}

    \begin{columns}[T,onlytextwidth]
        \column{0.48\textwidth}
        \textbf{Inputs:}
        {\small Interchannel level and phase difference features from
            \[ \footnotesize
            R[f] = \timeavg \frac{X_2[f,t]}{X_1[f,t]}
            \approx \timeavg \frac{H_2[f]\alert{S[f,t]}}{H_1[f]\alert{S[f,t]}}
            \]
            \\$\approx$ the relative transfer function
            \\$\to$ remove source dependency
        }

        \column{0.48\textwidth}
        \textbf{Output:} {\small Inter and intra arrival delays}
        \begin{columns}
            \column{0.60\textwidth}
            \includegraphics<1>[width=\textwidth]{figures/lantern_rir_tdoa1(1).png}%
            \includegraphics<2>[width=\textwidth]{figures/lantern_rir_tdoa1(2).png}%
            \includegraphics<3->[width=\textwidth]{figures/lantern_rir_tdoa1(3).png}%
            \column{0.45\textwidth}
            \footnotesize
            \centering
            4 TOA
            \\$\downarrow$
            \\3 Time \alert{Difference} of Arrivals (\alert{\textbf{TDOAs}})
        \end{columns}
        \begin{center}
            \small
            \textbf{HP:} first $\Leftrightarrow$ strongest echo
        \end{center}
    \end{columns}

    \pause[4]
    \vfill
    \begin{itemize}
        \item Architecture: $\MLP$, $\CNN$~{\footnotesize\cite{chakrabarty2017broadband,nguyen2018autonomous}}
        \item Loss Function:
        \begin{enumerate}
            \item RMSE (Multi-label regression) $\to$ TDOAs
            \item Gaussian log-likelihood $\to \kbrace{\mu_\tau, \sigma^2_\tau} \;\forall \tau \in$ TDOAs \hspace{1em}\tikzmark{CNN}\tikzmark{CNNtop}
            \item Student log-likelihood $\to \kbrace{\mu_\tau, \lambda_\tau, \nu_\tau} \;\forall \tau \in$ TDOAs \tikzmark{CNNbot}
        \end{enumerate}

        \pause[6]
        \item Data:
        \begin{itemize}
            \item synthetic data
            \item white-noise as source signal + AWGN of 0, 10, 20 dB
            \item 2 microphone in close-surface scenario
        \end{itemize}
    \end{itemize}

    % \visible<5->{
    % \begin{tikzpicture}[overlay, remember picture]
    %     \node[anchor=base] (a) at (pic cs:CNNtop) {\vphantom{h}}; % push the mark to the top of the line (ie including ascenders)
    %     \node[anchor=base] (b) at (pic cs:CNNbot) {\vphantom{g}}; % push the mark to the bottom of the line (ie including descenders)
    %     \draw [decoration={brace,amplitude=0.35em},decorate,thick,gray]
    %         (a.north -| {pic cs:CNN}) -- node[right,inner sep=1em] {
    %             \small \parbox{12em}{Good for data fusion\\Similar to \textbf{MDN}\\~{\footnotesize\cite{bishop1994mixture}}}
    %             } (b.south -| {pic cs:CNN});
    % \end{tikzpicture}
    % }
    % \visible<1->{
    %     \footnotetext[1]{\tiny $\mathtt{ILD} = \log\kvbar{R}$, $\mathtt{IPD} = \arg{R/\kvbar{R}}$}
    % }

\end{frame}

\begin{frame}{\faFlask~Experimental results \hfill\faProjectDiagram}

    \vspace{2mm}
    \begin{mycontriblock}
        \textbf{Proposed Method:} $\MLP$, $\CNN$, $\CNNn$, $\CNNt$
    \end{mycontriblock}
    \begin{mysotablock}
        \textbf{Baseline:} $\GCCPHAT${\tiny~\cite{knapp1976generalized}}
    \end{mysotablock}

    \textbf{Metrics:} normalized RMSE (0 = best fit, 1 = random fit)

    % \vspace{2mm}
    % \begin{columns}[T,onlytextwidth]

    %     \column{0.32\textwidth}
    %     Are better than baseline?
    %     \\\addendum{\footnotesize only TDOA on direct path}
    %     \\\hspace{.3em}$\to$ yes \cmark

    %     \column{0.32\textwidth}
    %     Echoes' \alert{TDOAs}?
    %     \\\hspace{.3em}$\to$ yes \cmark
    %     \\\hspace{.3em}$\to$ CNN better than MLP

    %     \column{0.32\textwidth}
    %     Are robust to noise?
    %     \\\hspace{.3em}$\to$ yes \cmark
    %     \\\hspace{.3em}$\to$ \CNNn/\CNNt better than \CNN

    % \end{columns}

    \begin{center}
        \includegraphics<1>[trim={0 30 0 0},clip,height=0.33\textwidth]{figures/lantern_snr1.pdf}%
        \includegraphics<2>[trim={0 30 0 0},clip,height=0.33\textwidth]{figures/lantern_snr2.pdf}%
        \includegraphics<3>[trim={0 30 0 0},clip,height=0.33\textwidth]{figures/lantern_snr3.pdf}%
        \includegraphics<4->[trim={0 30 0 0},clip,height=0.33\textwidth]{figures/lantern_snr4.pdf}%
    \end{center}

    \vspace{-3mm}
    \visible<2->{\textbf{Observation:}}

    \vspace{-2mm}
    \begin{itemize}\small
        \item<2->[\cmark] $\MLP$ outperforms $\GCCPHAT$ on TDOA estimation
        \item<3->[\cmark] $\CNN$ outperforms $\MLP$ (lower error and smaller variance)
        \item<4->[\cmark] $\CNNn$ and $\CNNt$ outperform $\CNN$ (lower error and smaller variance)
        \item<4->[\xmark] TDOA between DP and 1$^\text{st}$ echo more difficult
        \item<5->[\xmark] In general, only the first echo on white noise
    \end{itemize}

\end{frame}


\subsection{\blaster}

\begin{frame}{(Discrete) RIR-based methods: the State of the Art \hfill\faBook}
    \small
    \begin{block}{Key ingredient -- \textit{Cross relation identity}}
        \begin{equation*}
            \begin{cases}
                \contMic_1 &= \contRIR_1 \contConv \contSrc\\
                \contMic_2 &= \contRIR_2 \contConv \contSrc
            \end{cases}
        \end{equation*}

        \begin{equation*}
            \contRIR_2 \contConv \contMic_1 = \textcolor{gray}{\contRIR_2 \contConv \contRIR_1 \contConv \contSrc}
            \textcolor{gray}{= \contRIR_1 \contConv \contRIR_2 \contConv \contSrc} = \contRIR_1 \contConv \contMic_2
        \end{equation*}
    \end{block}

    \begin{block}{Ideas:}
    \begin{enumerate}
        \small
        % \item Sampled version of $\contMic_1,\contMic_2$ are available: $\discMic_1, \discMic_2$ WRITE EVERYTHING IN THE DISCRETE
        \item Echo TOAs $\propto$ sampling frequency
        \item Find echoes $\rightarrow$ \textbf{find sparse non-negative vectors} $\discRIR_1, \discRIR_2$ of length $L$
        \item Modeled as \textbf{Lasso}-like problem

        \vspace*{2mm}
        \begin{mysotablock}
            \begin{equation*}
                \widehat{\discRIR}_1, \widehat{\discRIR}_2 \in
                \underset{\discRIR_1, \discRIR_2\in\kR^n}{\arg\min}\;
                \Vert \discMic_1 \contConv \discRIR_2 - \discMic_2 \tikzmarknode{conv}{\contConv} \discRIR_1 \Vert_2^2
                + \lambda \mathcal{P}(\discRIR_1, \discRIR_2)
                \quad\text{s.t.}\quad\mathcal{C}(\discRIR_1, \discRIR_2)
            \end{equation*}

            \vspace*{-2mm}
            \begin{center}
                \footnotesize
                $\mathcal{P}(\discRIR_1, \discRIR_2)$ $\longrightarrow$ sparse promoting regularizer
                \hspace{5mm} \footnotesize $\mathcal{C}(\discRIR_1, \discRIR_2)$ $\longrightarrow$ constraints e.g. \parbox{6em}{nonnegativity\\anchor}
            \end{center}
        \end{mysotablock}
        \begin{tikzpicture}[overlay,remember picture, %
            nodes={inner sep=1pt, align=center, color=gray, font=\footnotesize}, %
            gray,>=stealth] %
            \draw[->] (conv.north) to[out=90, in=180] ++ (+10mm,+4mm) node[right] %
            {{= $\mathtt{Toeplitz}(\discMic_i) \discRIR_j \in \mathcal{O}(L^2)$}};
        \end{tikzpicture}
    \end{enumerate}
    \end{block}

    \vspace{-12mm}
    \begin{block}{}
        \begin{center}
            \footnotesize
            \textcolor{mygreen}{\cmark}  \cite{tong1994blind} \qquad \textcolor{mygreen}{\cmark}  \cite{lin2008blind} \qquad \textcolor{mygreen}{\cmark} \cite{aissa2008blind} \\
            \textcolor{mygreen}{\cmark} \cite{kowalczyk2013blind} \qquad \textcolor{mygreen}{\cmark} \cite{crocco2016estimation}
        \end{center}
    \end{block}

 \end{frame}


\begin{frame}{Proposed approach: analytical \& off-grid \hfill\faJediOrder}

    {\hfill \faPeopleCarry~C. Elvira.}

    \begin{block}{\textbf{Observation 1:} the cross-relation remains true in the \alert{continuous} frequency domain}
        \begin{equation*}
            \mathcal{F}x_1 \cdot \mathcal{F}h_2 (\sfrac{n}{F_s}) = \mathcal{F}x_2 \cdot \mathcal{F}h_1(\sfrac{n}{F_s}) \qquad n=0\dots N-1
        \end{equation*}
        \end{block}

        \vspace{.5em}

        \pause
        \begin{block}{\textbf{Observation 2:} $\mathcal{F}\delta_{\mathrm{echo}}$ is known in \alert{closed-form}}
        \end{block}

        \pause
        \vspace{1.em}
        \begin{block}{\textbf{Observation 3:} $\mathcal{F}{\mathrm{x_i}}$ can be (well) approximated by \alert{DFT}}
        \begin{equation*}
            \mathbf{X}_i = \texttt{DFT}(\discMic_i) \simeq  \mathcal{F}{\contMic_i}(nF_s) \qquad n=0\dots N-1
        \end{equation*}
        \end{block}


        \pause
        \vfill
        \setbeamercolor{block title}{fg=white,bg=darkblue}
        \setbeamercolor{block body}{fg=black,bg=bluegreen!10}
        \begin{block}{\textbf{Idea:} Recover echoes by matching a finite number of frequencies}
        \begin{equation*}
            \underset{h_1,h_2 \in \substack{\text{measure} \\ \text{space}}}{\arg\min} \;
            \tfrac{1}{2} \kvvbar{
                \mathbf{X}_1 \cdot \mathcal{F}h_2 (f) - \mathbf{X}_2 \cdot \mathcal{F}h_1(f)
            }_2^2
            + \lambda \kvvbar{h_1 + h_2}_{\mathrm{TV}}
            \quad
            \text{s.t.}\;
            \begin{cases}
                h_1(\{0\})=1 \\
                h_l \geq 0
                \end{cases}
        \end{equation*}
        \end{block}

        \begin{center}
            \small
            $\sim$ \textbf{Lasso}, but $\mathcal{F}h_2 (f)$ is a continuous function $\to$ \textbf{BLasso}~\cite{bredies2020sparsity}

            \textcolor{mygreen}{\cmark{No huge matrix} \qquad
            \cmark \, \parbox{8.5em}{Solutions is \\ a train of Dirac} \qquad
            \cmark \, \parbox{8em}{anchor prevents \\ trivial solution}}

            Perfect recovery in the noiseless synthetic case
            \cmark \pro{Promising results on noiseless data with RIRs matching the echo-model}
        \end{center}

        % \begin{center}
        %     solved with Sliding Frank-Wolfe algorithm \cite{denoyelle2019sliding}
        % \end{center}

\end{frame}

\begin{frame}[t]{\faFlask~Experimental results \hfill\faJediOrder}

    \begin{columns}[onlytextwidth]
        \begin{column}{0.50\textwidth}
            \begin{block}{Syntetic Dataset at 16 kHz}
                \small
                \begin{itemize}
                    \item 2 microphones, 1 sound source
                    \item Shoebox with random geometry
                    \item 2 signals: broadband and speech
                    \item $\mathcal{D^{\text{SNR}}}$: $SNR \in [0, 20]$ dB, $\text{RT}_{60} = 400$ ms
                    \item $\mathcal{D^{\text{RT60}}}$: $\text{RT}_{60} = [100, 1000]$ ms, $SNR = 20$ dB
                \end{itemize}
            \end{block}
        \end{column}

        \begin{column}{0.45\textwidth}
            \includegraphics[width=0.9\textwidth]{figures/aer_scenario4.png}
        \end{column}
    \end{columns}

    \begin{mysotablock}

        \textbf{Baselines:} discrete RIR-based methods based on LASSO
        \begin{itemize}
            \item BSN: Blind, Sparse and Non-negative \cite{lin2007blind}
            \item IL1C: iteratively-weighted $\ell_1$ constraint~\cite{crocco2015room}
            \\\hspace{.3em} $\hookrightarrow$ State of the Art
        \end{itemize}
        {\footnotesize \hfill hyperparameters and peak-picking tuned via cross-validation}
    \end{mysotablock}

    \vspace{2mm}
    \begin{mycontriblock}
        \textbf{Proposed method:} continuous RIR-agnostic based on BLasso
        \\Blind and Sparse Technique for Echo Retrieval (\blaster)
    \end{mycontriblock}

\end{frame}


\begin{frame}<1>[label=echoes]{Performance per \# of echoes \hfill\faJediOrder}

    \textbf{Metric:} \alert{Precision} = how many estimated echoes are correct (within 2 samples)

    \begin{center}
        \includegraphics[width=0.8\linewidth]{figures/p_k-7_thr-2_bns_crocco_blaster-peak_withRechoes.pdf}
        \\\addendum{$\text{RT}_{60} = 400$ ms and SNR = 20 dB.}
    \end{center}

    \begin{center}
        \textcolor{myred}{\xmark \: \parbox{8em}{Sensitive\\to \# echoes}}
        \textcolor{myred}{\xmark \: \parbox{8em}{Sensitive\\source signal}}
        \only<1>{
        \textcolor{mygreen}{\cmark \: \parbox{8em}{Good\\for 2 echoes}}
        }
        \only<2>{
        \textcolor{mygreen}{
        \cmark \: \parbox{8em}{Good or 2 echoes
            \\{\footnotesize\cite{scheibler2018separake,di2019mirage}}}}
        }
    \end{center}

\end{frame}

\againframe<2>{echoes}

\begin{frame}{Error per Dataset/Signal while recovering 7 echoes \hfill\faJediOrder}

    \textbf{Metric:} \alert{RMSE} on matched echoes = error on the correct guess

    \begin{center}
        \begin{overpic}[width=0.6\textwidth]{figures/e_k-7_thr-2_bns_crocco_blaster.pdf}
            \put (102, 48){\includegraphics[width=5em]{figures/legend.pdf}}
        \end{overpic}
        \\\addendum{Fs = 16 kHz}
    \end{center}

    \begin{center}
        \textcolor{mygreen}{\cmark{Lower RMSE}} \qquad
        \textcolor{mygreen}{\cmark \, \parbox{8.5em}{Robustness\\
        to SNR and $\text{RT}_{60}$}} \qquad
        \textcolor{myred}{\xmark \, \parbox{8em}{Source signal\\dependent}}
    \end{center}

\end{frame}


% % \subsection{Interim conclusion (2/4)}

% % \begin{frame}{Interim conclusion (2/4)}
% %     \begin{block}{on Acoustic Echo Retrieval:}
% %         \begin{itemize}
% %             \item Most of the literature is on Passive and RIR-based, with on-grid approaches
% %             \item On-grid approaches suffers by the off-grid nature of the echoes (complexity, sampling)
% %         \end{itemize}
% %     \end{block}

% %     \begin{block}{on \blaster:}
% %         \begin{itemize}
% %             \item[\cmark] off-grid parameter-free which exploit dirac closed-form model (non negativity and sparsity)
% %             \item[\cmark] smaller RMSE due to super-resolution, better for small \# of echoes
% %             \item[\xmark] source dependent and on number of echoes
% %             \item[\xmark] validate only on synthetic data
% %             \item[$\rightarrow$] Multichannel and RTF-based extention
% %         \end{itemize}
% %     \end{block}

% %     \begin{block}{on \lantern:}
% %         \begin{itemize}
% %             \item[\cmark] promising results for first echo estimation
% %             \item[\cmark] direct application for table top application
% %             \item[\xmark] difficult extention
% %             \item[\xmark] need for real data validation
% %             \item[$\rightarrow$] physically-constrained neural network
% %             \item[$\rightarrow$] missing frequencies in the input
% %         \end{itemize}
% %     \end{block}
% % \end{frame}