\subsection{From Physics to Digital Signal Processing}

\begin{frame}{Echoes and Room Acoustics}

    \begin{block}{Sound propagates and interacts with space}
        \begin{itemize}
            \item it \alert{travels} with a certain speed and it is \alert{attenuated};
            \item it is \alert{absorbed} and \alert{reflected} by surfaces;
            \item and it is scattered, diffracted, etc.
        \end{itemize}
    This is describe by the so called RIRs
    \end{block}

    \begin{figure}
        \centering
        \includegraphics[width=0.3\textwidth]{example-image-a}
    \end{figure}

    \begin{block}{Elements of reverberation}
        \cite{Shroeder, Allen, Polack, Kuttruff}
        \begin{itemize}
            \item Direct path
            \item \alert{Early Echoes}
            \item Reverberation tails
        \end{itemize}

    \end{block}


    \begin{mydefblock}{Early Echoes}
        Reflection
        \\(perceptually and physically motivated distinction)
    \end{mydefblock}

\end{frame}

\begin{frame}{Echoes and Room Impulse Response}

    \begin{block}{RIRs can be modeled with the Image Methods}
        \begin{itemize}
            \item specular reflection only
            \item ``playing billiard in a concert hall''
            \item for shoebox room it is is the solution for physics
            \item in frequency domain it writes as
        \end{itemize}
    \end{block}

    \begin{block}{RIRs accounts for}
        the \alert{geometry} of the room
        \begin{itemize}
            \item Room shape and size
            \item Mic and Source position
            \item presence of objects
        \end{itemize}
        the acoustic properties of the audio scene
        \begin{itemize}
            \item surface materials
            \item objects materials
        \end{itemize}
    \end{block}

    examples
    \begin{figure}
        \centering
        \includegraphics[width=0.3\textwidth]{example-image-a}
    \end{figure}

\end{frame}

\begin{frame}{Echoes in (Digital) Signal Processing}

    \begin{block}{Room Impulse Response}
        \begin{equation*}
            \tilde{x}_i = (\tilde{h}_i \ast \tilde{s})(t) \longrightarrow \tilde{X}_i( f) = \tilde{H}_{ij}( f) \tilde{S}( f)
        \end{equation*}
        the linear filtering effect due to the propagation of sound from a source to a microphone in a indoor space
    \end{block}

    \begin{block}{Observation}
        Our vision is limited both in time (finite and discrete) and in frequency (finite and discrete)
        \begin{equation}
            x_i[n] = ...
        \end{equation}
    \end{block}

    \begin{block}{Signal model in the frequency domain}
        \begin{equation*}
            x_i = (h_i \ast s)(t)\;\longrightarrow\;X(f) = H_i(f) S(f)
        \end{equation*}
    \end{block}

    \begin{block}{Approximations}
        \begin{itemize}
            \item Narrowband Approximation
            \item DTFT echo model in the DFT
        \end{itemize}
    \end{block}

\end{frame}

\subsection*{interim conclusion}
\begin{frame}{Interim Conclusion I}
    \begin{alertblock}{Approximations}
        \begin{itemize}
            \item Echoes are well described by specular reflection
            \item Echoes are off-grid by nature
            \item Sampling and quantization make them hard
            \item Processing in the discrete frequency domain, but with continuous time echo model
        \end{itemize}
    \end{alertblock}
\end{frame}
